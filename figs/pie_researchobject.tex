
%\subsection{Pie chart for Research Object (75)}
%\begin{figure}
\begin{center}
\begin{tikzpicture}[scale=2]
\pgfmathsetcounter{pieb}{0}
\foreach \p/\q/\t/\c in {21/16/Architecture Analysis Method/blue!20, 16/12/Architecture Design Method/blue!30, 11/8/Architecture Optimization Method/blue!40, 9/7/Architecture Evolution/blue!50, 9/7/Architecture Description Language/blue!60, 5/4/Architecture Decision Making/blue!70, 5/4/Reference Architecture/blue!80, 4/3/Architecture Pattern/blue!90, 4/3/Architecture Description/blue!100, 4/3/Architectural Aspects/blue!110, 4/3/Teaching/blue!120, 3/2/Technical Dept/blue!130, 1/1/Quality Evolution/blue!140, 1/1/Architecture Extraction/blue!150, 1/1/Architectural Assumptions/blue!160}
  {
    \setcounter{piea}{\value{pieb}}
    \addtocounter{pieb}{\q}
    \slice{\thepiea/75*360}
          {\thepieb/75*360}
          {\p\%}{\t}{\c}
  }
\end{tikzpicture}
\textbf{Pie chart for Research Object (75)}
\end{center}
%\caption{Pie chart for Research Object (75)}
%\label{fig:pie_researchobject}
%\end{figure}

